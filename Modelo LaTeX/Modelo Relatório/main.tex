%! TEX root = main.tex
\documentclass[
  12pt,  % font-size
  oneside,  % Format every page equal
  a4paper,  % Paper
  %  ABNTex2 settings. TITLE is uppercase
  chapter=TITLE, 
  section=TITLE,
  %subsection=TITLE,
  %subsubsection=TITLE,
  % Babel settings
  english,
  brazil
]{abntex2}

\usepackage{setup/setup}

%------------------------------

% ---
%   Bibliography setting
% ---

\usepackage[backend=biber, style=numeric-comp]{biblatex}  % style=numeric-comp for numeric style

\setlength\bibitemsep{\baselineskip}
\DeclareFieldFormat{url}{Disponível~em:\addspace\url{#1}}
\NewBibliographyString{sineloco}
\NewBibliographyString{sinenomine}
\DefineBibliographyStrings{brazil}{%
	sineloco     = {\mkbibemph{S\adddot l\adddot}},
	sinenomine   = {\mkbibemph{s\adddot n\adddot}},
	andothers    = {\mkbibemph{et\addabbrvspace al\adddot}},
	in			 = {\mkbibemph{In:}}
}

\addbibresource{aftertext/references.bib}

\DeclareSourcemap{
  \maps[datatype=bibtex]{
    \map{
      \step[fieldset=abstract, null]
      \step[fieldset=pagetotal, null]
    }
    \map{
      \pertype{inproceedings}
			\step[fieldset=venue, null]
			\step[fieldset=eventdate, null]
			\step[fieldset=eventtitle, null]
			\step[fieldset=isbn, null]
			\step[fieldset=volume, null]
    }
  }
}

%------------------------------


% ---
%   Cover informations
% ---

\autor{Nome completo do autor(a)}

\titulo{Título do trabalho}

\subtitulo{Substitulo (se houver)}

\orientador{Prof. XXXXX, Dr.}
%\orientador[Orientadora]{Prof. XXXXX, Dra.}

\coorientador{Prof. XXXXX, Dr.}
%\coorientador[Orientadora]{Prof. XXXX, Dra.}

\coordenador{Prof. XXXX, Dr.}
%\coodernador[Coordenadora]{Prof. XXXX, Dra.}

\ano{Ano}
\data{[dia] de [mês] de [ano]}

\local{cidade}

\instituicaosigla{UFSC}
\instituicao{Universidade Federal de Santa Catarina}

\tipotrabalho{Relatório projeto}

\formacao{[licenciado/bacharel] em [nome do titulo obtido]}

\nivel{[licenciado/bacharel]}

\programa{Curso de Graduação em [XXXXXX]}

\centro{Campus [XXXXX] Depertamento de [XXXXX]}

\preambulo
{
  \imprimirtipotrabalho~do~\imprimirprograma~da\imprimirinstituicao~para~a~obtenção~do~título~de~\imprimirformacao.
}

%------------------------------


% ---
%   PDF settings
% ---
\definecolor{blue}{RGB}{41,5,195}
\makeatletter
\hypersetup{
		pdftitle={\@title}, 
		pdfauthor={\@author},
    pdfsubject={\imprimirpreambulo},
    pdfcreator={LaTeX with abnTeX2},
		pdfkeywords={ufsc, latex, abntex2}, 
		colorlinks=true,
    linkcolor=black,
    citecolor=black,
    filecolor=black,
		urlcolor=black,
		bookmarksdepth=4
}
\makeatother


% ---
%   Symbols and Acronym
% ---

% Acronym declaration
\siglalista{ABNT}{Associação Brasileira de Normas Técnicas}

% Symbol declaration
\simbololista{C}{\ensuremath{C}}{Circunferência de um círculo}
\simbololista{pi}{\ensuremath{\pi}}{Número pi} 
\simbololista{r}{\ensuremath{r}}{Raio de um círculo}
\simbololista{A}{\ensuremath{A}}{Área de um círculo}

% Compile acronym and symbols
\makenoidxglossaries

% Compile index
\makeindex

%------------------------------

%%%%%%%%%%%%%%%% DOCUMENT STARTS HERE %%%%%%%%%%%%%%%%

\begin{document}
  \selectlanguage{brazil}
  \frenchspacing
  \OnehalfSpacing


  % ---
  %   ELEMENTOS PRÉ-TEXTUAIS
  % ---
  %   Capa, folha de rosto, ficha bibliografica, errata, folha de aprovação, dedicatória, agradecimentos, epígrafe, resumos, lista
  % Capa
\imprimircapa


% Folha de Rosto
\imprimirfolhaderosto

% Resumo em Português 
\setlength{\absparsep}{18pt} % ajusta o espaçamento dos parágrafos do resumo
\begin{resumo}
    \SingleSpacing
    AA
    \textbf{Palavras-chave}: Nanomateriais. Cristalografia. Mecanoquímica. 
\end{resumo}

% Resumo em Inglês 
%\begin{resumo}[Abstract]
%	\SingleSpacing
%	\begin{otherlanguage*}{english}
%		Resumo traduzido para outros idiomas, neste caso, inglês. Segue o formato do resumo feito na língua vernácula. As palavras-chave traduzidas, versão em língua estrangeira, são colocadas abaixo do texto precedidas pela expressão “Keywords”, separadas por ponto.
%		
%		\textbf{Keywords}: Keyword 1. Keyword 2. Keyword 3.
%	\end{otherlanguage*}
%\end{resumo}

{%hidelinks
	\hypersetup{hidelinks}
 
	\imprimirlistadesiglas
        
	% inserir o sumario
	\pdfbookmark[0]{\contentsname}{toc}
	\tableofcontents*
	\cleardoublepage
	
}%hidelinks



  % ---
  %   ELEMENTOS TEXTUAIS
  % ---
  \textual

  % 1. Introdução
  \include{chapters/1-chapter}

  % n. Conclusão
  \include{chapters/conclusao}


  % ---
  %   ELEMENTOS PÓS-TEXTUAIS
  % ---
  \postextual

  % Referências bibliograficas
  \begingroup
    \SingleSpacing\printbibliography[title=REFERÊNCIAS]
  \endgroup

  % Glossário
%  \glossary

  % Apendices
  \begin{apendicesenv}
    \input{aftertext/apendice_a}
  \end{apendicesenv}

  % Anexos
  \begin{anexosenv}
    \input{aftertext/anexo_a}
  \end{anexosenv}

  % Indice Remissivo
  %\phantompart
  %\printindex

\end{document}
